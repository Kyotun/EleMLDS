% Teaching content definitions.
% Use \DefineTeach{<number>}{...} where <number> matches the section numbering.
% Examples: 1, 1.2, 1.2.1

% 1) INTRODUCTION TO DATA SCIENCE
% 1.2 Tabular Data
%
% Use the structure below when adding teaching content.
% Note: \DefineTeach{1.2} is just an example to illustrate the format.
% \DefineTeach{1.2}{%
% % Write teaching content for "1.2 Tabular Data" here.
%
% % Template (copy and adapt):
% % \begin{examlikelihood}{Medium}
% % Why this appears on exams and how to recognize it.
% % \end{examlikelihood}
% %
% % \begin{examfavorite}
% % What instructors love to ask and typical phrasing.
% % \end{examfavorite}
% %
% % Why (motivation): ...
% % What (definition): ...
% % How (procedure/usage): ...
% %
% % \begin{cheatsheet}
% % \begin{itemize}
% %   \item Must‑memorize point 1
% %   \item Must‑memorize point 2
% % \end{itemize}
% % \end{cheatsheet}
% %
% % \begin{pitfall}
% % Common mistake and how to avoid it.
% % \end{pitfall}
% %
% % \begin{visualbox}
% % \centering
% % \begin{tikzpicture}[]
% % % Simple diagram
% % \end{tikzpicture}
% % \end{visualbox}
% %
% % \textbf{Key takeaways:} ...
% % }
